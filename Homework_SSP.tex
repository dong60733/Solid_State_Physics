\documentclass[11pt]{ctexart}

% 插入宏包
\usepackage{graphicx} % Required for inserting images
\usepackage{geometry} % 设置页边距
\usepackage{lipsum} % 生成虚拟文本
\usepackage{fancyhdr} 
\setlength{\headheight}{14pt}% 自定义页眉和页脚
\usepackage{booktabs} % 插入三线表
\usepackage{lastpage} % 解决总页数显示问题
\usepackage{amsmath,amsfonts,amsthm} % 常用数学公式指令、数学公式、提供证明环境
\usepackage{bm} % 数学字体加粗
\usepackage{mathrsfs} % 提供特殊的数学花体
\usepackage{amssymb, ,hyperref, framed, color, enumerate}
\usepackage{bbding} % 打叉、打勾

% Custom counter for problems
\newcounter{problemname}

% Environment for problems
\definecolor{shadecolor}{RGB}{241, 241, 255}

\newenvironment{problem}{\begin{shaded}\stepcounter{problemname}\par\noindent\textbf{习题}\arabic{problemname}.}{\end{shaded}\par}


% Environment for solutions
\newenvironment{solution}{\par\noindent\textbf{解答. }}{\par}

% Environment for notes
\newenvironment{note}{\par\noindent\textbf{习题\arabic{problemname}的注记. }}{\par}

% Reset problem counter at each subsection
\usepackage{titlesec}
\titleformat{\subsection}{\normalfont\large\bfseries}{\thesubsection}{1em}{\setcounter{problemname}{0}}

% 设置文章格式
\geometry{left=2.5cm,right=2.5cm,top=3cm,bottom=3cm}
\pagestyle{fancy} % 使用fancyhdr宏包定义页眉页脚
\fancyhf{} % 清空默认的页眉和页脚设置
\linespread{1.5}
\chead{《固体物理学(胡安版)》作业}
\cfoot{第 \thepage 页(共 \pageref{LastPage}页)}

% 信息栏
\title{\Huge\textbf{固体物理学作业}}
\author{Charles Luo}
\date{\today}

% 正文区
\begin{document}
\maketitle
\newpage
\tableofcontents
\newpage

\section{第一章习题}

\begin{problem}
    在正交直角坐标系中,若矢量 $\mathbf{R}_n = n_1\,\mathbf{i} + n_2\,\mathbf{j} + n_3\,\mathbf{k}$, 其中 $\mathbf{i},\,\mathbf{j},\,\mathbf{k}$
    为单位矢量,$n_i\ (i = 1,2,3)$ 为整数。问下列情况属于什么点阵?
    \begin{enumerate}[(a)]
        \item 当 $n_i$ 为全奇加全偶时;
        \item 当 $n_i$ 之和为偶数时。
    \end{enumerate}
\end{problem}
\begin{solution}

\end{solution}

\begin{problem}
    分别证明:
    \begin{enumerate}[(a)]
        \item 面心立方(fcc)和体心立方(bcc)点阵的惯用初基元胞三基矢间夹角 $\theta$ 相等,
              对fcc为 $60^\circ$ ,对bcc为 $109^\circ27^\prime$;
        \item 在金刚石结构中,作任一原子与其四个最近邻原子的连线。证明任意两条线之间夹角 $\theta$ 均为 \\[12pt]
              $\arccos\left(-\dfrac{1}{3}\right) = 109^\circ27^\prime$。
    \end{enumerate}
\end{problem}
\begin{solution}
    
\end{solution}

\begin{problem}
    证明在六角晶系中米勒指数为 $(hkl)$ 的晶面族间距为
    $$d = \left[\frac{4}{3}\left(\frac{h^2 + hk + k^2}{a^2} + \frac{l^2}{c^2}\right)\right]^{-\frac{1}{2}}.$$
\end{problem}
\begin{solution}
    
\end{solution}

\begin{problem}
    证明底心正交点阵的倒点阵仍为底心正交点阵。
\end{problem}
\begin{solution}
    
\end{solution}

\begin{problem}
    试证明具有四面体对称性的晶体,其介电常量为一标量介电常量:
    $$\varepsilon_{\alpha\beta} = \varepsilon_0\delta_{\alpha\beta}.$$
\end{problem}
\begin{solution}
    
\end{solution}

\begin{problem}
    若 $AB_3$ 的立方结构如图所示,设 $A$ 原子的散射因子为 $f_A(\mathbf{K}_{hkl})$,
    $B$ 原子的散射因子 $f_B(\mathbf{K}_{hkl})$.
\end{problem}
\begin{solution}
    
\end{solution}

\begin{problem}
    在某立方晶系的铜 $\mathbf{K}_\alpha X$ 射线粉末相中,观察到的衍射角 $\theta_i$ 有下列关系:
    $$\sin{\theta_1} : \sin{\theta_2} : \sin{\theta_3} : \sin{\theta_4} 
    : \sin{\theta_5}: \sin{\theta_6}: \sin{\theta_7}: \sin{\theta_8}$$
    $$= \sqrt{3} : \sqrt{4} : \sqrt{8} : \sqrt{11} : \sqrt{12} : \sqrt{16} : \sqrt{19} : \sqrt{20}.$$
\end{problem}
\begin{solution}
    
\end{solution}

\begin{problem}
    X 射线衍射的线宽。 \\
    假定一个有限大小的晶体,点阵节点由 $\displaystyle R_l = \sum_{i = 1}^{3}l_i\mathbf{a}_i$ 确定,
    其中 $l_i$ 取整数 $0,1,2,\cdots,N_i - 1$,每个结点处有全同的点散射中心。散射振幅可写为
    $$u_{k\to k^\prime} = c\sum_{l_i = 0}^{N_i - 1}\text{e}^{-i\left(k^\prime - k\right)\cdot\sum\limits_{i = 1}^{3}l_i\mathbf{a}_i}.$$
    \begin{enumerate}[(a)]
        \item 证明散射强度 $\displaystyle I = \left|u\right|^2 = u^*u = c^2\prod_{i = 1}^{3}
              \dfrac{\sin^2{\dfrac{1}{2}N_i\left(\Delta\mathbf{k}\cdot\mathbf{a}_i\right)}}{\sin^2{\dfrac{1}{2}\left(\Delta\mathbf{k}\cdot\mathbf{a}_i\right)}},\ \Delta k = k^\prime - k$;
        \item 当 $\Delta\mathbf{k}\cdot\mathbf{a}_i = 2\pi h_i$($h_i$ 为整数)时,出现衍射极大值,函数 $\sin^2{\dfrac{1}{2}N_i\left(\Delta\mathbf{k}\cdot\mathbf{a}_i\right)}$ 的第一个零点定义了 X 射线衍射的线宽
              $\Delta_i$,证明 $\Delta_i = \dfrac{2\pi}{N_i}$;
        \item 对于一个无限大的晶体,$\displaystyle N_i\to\infty$,证明 $\displaystyle I = c^2N^2\delta_{\mathbf{k}^\prime - \mathbf{k},\mathbf{K}_h}$.
    \end{enumerate}
\end{problem}
\begin{solution}

\end{solution}

\newpage
\section{第二章习题}

\end{document}

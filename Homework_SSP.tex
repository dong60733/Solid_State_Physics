\documentclass[11pt]{ctexart}

% 插入宏包
\usepackage{graphicx} % Required for inserting images
\usepackage{geometry} % 设置页边距
\usepackage{lipsum} % 生成虚拟文本
\usepackage{fancyhdr} 
\setlength{\headheight}{14pt}% 自定义页眉和页脚
\usepackage{booktabs} % 插入三线表
\usepackage{lastpage} % 解决总页数显示问题
\usepackage{amsmath,amsfonts,amsthm} % 常用数学公式指令、数学公式、提供证明环境
\usepackage{bm} % 数学字体加粗
\usepackage{mathrsfs} % 提供特殊的数学花体
\usepackage{amssymb, ,hyperref, framed, color, enumerate}
\usepackage{bbding} % 打叉、打勾

% Custom counter for problems
\newcounter{problemname}

% Environment for problems
\definecolor{shadecolor}{RGB}{241, 241, 255}

\newenvironment{problem}{\begin{shaded}\stepcounter{problemname}\par\noindent\textbf{习题}\arabic{problemname}.}{\end{shaded}\par}


% Environment for solutions
\newenvironment{solution}{\par\noindent\textbf{解答. }}{\par}

% Environment for notes
\newenvironment{note}{\par\noindent\textbf{习题\arabic{problemname}的注记. }}{\par}

% Reset problem counter at each subsection
\usepackage{titlesec}
\titleformat{\subsection}{\normalfont\large\bfseries}{\thesubsection}{1em}{\setcounter{problemname}{0}}

% 设置文章格式
\geometry{left=2.5cm,right=2.5cm,top=3cm,bottom=3cm}
\pagestyle{fancy} % 使用fancyhdr宏包定义页眉页脚
\fancyhf{} % 清空默认的页眉和页脚设置
\linespread{1.5}
\chead{《固体物理学(胡安版)》作业}
\cfoot{第 \thepage 页(共 \pageref{LastPage}页)}

% 信息栏
\title{\Huge\textbf{固体物理学作业}}
\author{Charles Luo}
\date{\today}

% 正文区
\begin{document}
\maketitle
\newpage
\tableofcontents
\newpage

\section{第一章习题}

\begin{problem}
    在正交直角坐标系中,若矢量 $\mathbf{R}_n = n_1\,\mathbf{i} + n_2\,\mathbf{j} + n_3\,\mathbf{k}$, 其中 $\mathbf{i},\,\mathbf{j},\,\mathbf{k}$
    为单位矢量,$n_i\ (i = 1,2,3)$ 为整数。问下列情况属于什么点阵?
    \begin{enumerate}[(a)]
        \item 当 $n_i$ 为全奇加全偶时;
        \item 当 $n_i$ 之和为偶数时。
    \end{enumerate}
\end{problem}
\begin{solution}
    \begin{enumerate}[(a)]
        \item 据题意,全奇加全偶应是两个布拉维格子的叠加。 \\
              若 $n_i\ (i = 1,2,3)$ 全为偶数,可以提取公因子 $2$ 得到 $\mathbf{R}_n = n_1^\prime\left(2\mathbf{i}\right) + n_2^\prime\left(2\mathbf{j}\right) + n_3^\prime\left(2\mathbf{k}\right)$,
              此时 $n_i^\prime\ (i = 1,2,3)$ 为整数,对应简单立方点阵,格矢长度为 $2$ 个单位长度。 \\
              同理可得 $n_i\ (i = 1,2,3)$ 全为奇数时也为简单立方点阵,可由全为偶数时点阵沿 $\left(1,1,1\right)$ 方向平移移一个单位长度得到。 \\
              二者的嵌套为体心立方点阵。
        \item 据题意,$n_1 + n_2 + n_3 = 2k,\ k\in\mathbb{N}$。 \\
              不妨取 $k = 1$ 和 $k = 2$ 来猜测,可以得到格点坐标为 $\left(1,1,0\right),\cdots,\left(0,1,1\right),\left(2,0,0\right),\cdots,\left(0,0,2\right)$.
              为面心立方点阵。
    \end{enumerate}
\end{solution}
\begin{note}
    \begin{itemize}
        \item (b)可取 $k_1 = k - n_1, k_2 = k - n_2, k_3 = k - n_1 - n_2$, 
                 得到 $\mathbf{R}_n = \left(k_2 + k_3\right)\mathbf{i} + \left(k_3 + k_1\right)\mathbf{j} + \left(k_1 + k_2\right)\mathbf{k}$.
    \end{itemize}
\end{note}

\begin{problem}
    分别证明:
    \begin{enumerate}[(a)]
        \item 面心立方(fcc)和体心立方(bcc)点阵的惯用初基元胞三基矢间夹角 $\theta$ 相等,
              对fcc为 $60^\circ$ ,对bcc为 $109^\circ27^\prime$;
        \item 在金刚石结构中,作任一原子与其四个最近邻原子的连线。证明任意两条线之间夹角 $\theta$ 均为 \\[12pt]
              $\arccos{\left(-\dfrac{1}{3}\right)} = 109^\circ27^\prime$。
    \end{enumerate}
\end{problem}
\begin{solution}
    \begin{enumerate}[(a)]
        \item fcc 三个基矢为 $\mathbf{a_1} = \left(0,\dfrac{1}{2},\dfrac{1}{2}\right),\mathbf{a_2} = \left(\dfrac{1}{2},0,\dfrac{1}{2}\right),\mathbf{a_3} = \left(\dfrac{1}{2},\dfrac{1}{2},0\right)$. \\[12pt]
              故 $\cos{\theta} = \dfrac{\mathbf{a_1}\cdot\mathbf{a_2}}{\left|\mathbf{a_1}\right|\left|\mathbf{a_2}\right|} = \dfrac{\mathbf{a_2}\cdot\mathbf{a_3}}{\left|\mathbf{a_2}\right|\left|\mathbf{a_3}\right|} = \dfrac{\mathbf{a_3}\cdot\mathbf{a_1}}{\left|\mathbf{a_3}\right|\left|\mathbf{a_1}\right|} = \dfrac{1}{2}$,即 $\theta = 60^\circ$. \\[12pt]
              bcc 三个基矢为 $\mathbf{a_1} = \left(-\dfrac{1}{2},\dfrac{1}{2},\dfrac{1}{2}\right),\mathbf{a_2} = \left(\dfrac{1}{2},-\dfrac{1}{2},\dfrac{1}{2}\right),\mathbf{a_3} = \left(\dfrac{1}{2},\dfrac{1}{2},-\dfrac{1}{2}\right)$. \\[12pt]
              故 $\cos{\theta} = \dfrac{\mathbf{a_1}\cdot\mathbf{a_2}}{\left|\mathbf{a_1}\right|\left|\mathbf{a_2}\right|} = \dfrac{\mathbf{a_2}\cdot\mathbf{a_3}}{\left|\mathbf{a_2}\right|\left|\mathbf{a_3}\right|} = \dfrac{\mathbf{a_3}\cdot\mathbf{a_1}}{\left|\mathbf{a_3}\right|\left|\mathbf{a_1}\right|} = -\dfrac{1}{3}$,即 $\theta = \arccos{\left(-\dfrac{1}{3}\right)} = 109^\circ27^\prime$.
        \item 金刚石结构中坐标为 $\left(\dfrac{1}{4},\dfrac{1}{4},\dfrac{1}{4}\right)$ 的原子相邻的4个原子坐标分别为 $\left(0,0,0\right),\left(0,\dfrac{1}{2},\dfrac{1}{2}\right),\left(\dfrac{1}{2},0,\dfrac{1}{2}\right),$ \\[12pt] 
              $\left(\dfrac{1}{2},\dfrac{1}{2},0\right)$. 邻边 $\mathbf{l_1} = \left(-\dfrac{1}{4},-\dfrac{1}{4},-\dfrac{1}{4}\right),\mathbf{l_2} = \left(-\dfrac{1}{4},\dfrac{1}{4},\dfrac{1}{4}\right),\mathbf{l_3} = \left(\dfrac{1}{4},-\dfrac{1}{4},\dfrac{1}{4}\right),\mathbf{l_4} = \left(\dfrac{1}{4},\dfrac{1}{4},-\dfrac{1}{4}\right)$. \\[12pt]
              故 $\cos{\theta} = \dfrac{\mathbf{l_1}\cdot\mathbf{l_2}}{\left|\mathbf{l_1}\right|\left|\mathbf{l_2}\right|} = \dfrac{\mathbf{l_2}\cdot\mathbf{l_3}}{\left|\mathbf{l_2}\right|\left|\mathbf{l_3}\right|} = \dfrac{\mathbf{l_3}\cdot\mathbf{l_4}}{\left|\mathbf{l_3}\right|\left|\mathbf{l_4}\right|} = \dfrac{\mathbf{l_4}\cdot\mathbf{l_1}}{\left|\mathbf{l_4}\right|\left|\mathbf{l_1}\right|} = -\dfrac{1}{3}$,即 $\theta = \arccos{\left(-\dfrac{1}{3}\right)} = 109^\circ27^\prime$.
    \end{enumerate}
\end{solution}

\begin{problem}
    证明在六角晶系中米勒指数为 $(hkl)$ 的晶面族间距为
    $$d = \left[\frac{4}{3}\left(\frac{h^2 + hk + k^2}{a^2} + \frac{l^2}{c^2}\right)\right]^{-\frac{1}{2}}.$$
\end{problem}
\begin{solution}
    米勒指数以单胞的三条棱为坐标系. \\[12pt]
    正点阵的一族晶面 $\left(hkl\right)$ 垂直于倒格矢 $\mathbf{K_h} = h\mathbf{b_1} + k\mathbf{b_2} + l\mathbf{b_3}$,晶面间距 $\dfrac{2\pi}{\left|\mathbf{K_h}\right|}$. \\[12pt]
    在六角晶系中 $\mathbf{a} = \left(a,0,0\right),\mathbf{b} = \left(-\dfrac{1}{2}a,\dfrac{\sqrt{3}}{2}a,0\right),\mathbf{c} = \left(0,0,c\right)$. \\[12pt]
    求倒点阵基矢:\\[12pt]
    $\mathbf{b_1} = 2\pi\dfrac{\mathbf{a_2}\times\mathbf{a_3}}{\mathbf{a_1}\cdot\left(\mathbf{a_2}\times\mathbf{a_3}\right)} = \dfrac{2\pi}{a}\left(1,\dfrac{\sqrt{3}}{3},0\right)$. \\[12pt]
    $\mathbf{b_2} = 2\pi\dfrac{\mathbf{a_3}\times\mathbf{a_1}}{\mathbf{a_1}\cdot\left(\mathbf{a_2}\times\mathbf{a_3}\right)} = \dfrac{2\pi}{a}\left(0,\dfrac{2\sqrt{3}}{3},0\right)$. \\[12pt]
    $\mathbf{b_3} = 2\pi\dfrac{\mathbf{a_1}\times\mathbf{a_2}}{\mathbf{a_1}\cdot\left(\mathbf{a_2}\times\mathbf{a_3}\right)} = \dfrac{2\pi}{c}\left(0,0,1\right)$. \\[12pt]
    倒格矢 $\mathbf{K_h} = \left(\dfrac{2\pi}{a}h,\dfrac{2\sqrt{3}\pi}{3a}h+\dfrac{4\sqrt{3}\pi}{3a}k,\dfrac{2\pi}{c}l\right)$, 故 $\displaystyle d = \dfrac{2\pi}{\left|\mathbf{K_h}\right|} = \left[\frac{4}{3}\left(\frac{h^2 + hk + k^2}{a^2} + \frac{l^2}{c^2}\right)\right]^{-\frac{1}{2}}.$
\end{solution}

\begin{problem}
    证明底心正交点阵的倒点阵仍为底心正交点阵。
\end{problem}
\begin{solution}
    底心正交阵基矢 $\mathbf{a_1} = \left(a,0,0\right),\mathbf{a_2} = \left(\dfrac{a}{2},\dfrac{b}{2},0\right),\mathbf{a_3} = \left(0,0,c\right)$. \\[12pt]
    倒点阵基矢: \\[12pt]
    $\mathbf{b_1} = 2\pi\dfrac{\mathbf{a_2}\times\mathbf{a_3}}{\mathbf{a_1}\cdot\left(\mathbf{a_2}\times\mathbf{a_3}\right)} = 2\pi\left(\dfrac{1}{a},-\dfrac{1}{b},0\right)$. \\[12pt]
    $\mathbf{b_2} = 2\pi\dfrac{\mathbf{a_3}\times\mathbf{a_1}}{\mathbf{a_1}\cdot\left(\mathbf{a_2}\times\mathbf{a_3}\right)} = 2\pi\left(0,\dfrac{2\pi}{b},0\right)$. \\[12pt]
    $\mathbf{b_3} = 2\pi\dfrac{\mathbf{a_1}\times\mathbf{a_2}}{\mathbf{a_1}\cdot\left(\mathbf{a_2}\times\mathbf{a_3}\right)} = 2\pi\left(0,0,\dfrac{1}{c}\right)$. \\[12pt]
    倒点阵仍为底心正交阵,底面边长为 $\dfrac{4\pi}{a}$ 和 $\dfrac{4\pi}{b}$,高为 $\dfrac{2\pi}{c}$.
\end{solution}

\begin{problem}
    试证明具有四面体对称性的晶体,其介电常量为一标量介电常量:
    $$\bm{\varepsilon}_{\alpha\beta} = \varepsilon_0\delta_{\alpha\beta}.$$
\end{problem}
\begin{solution}
    根据电动力学有
    $$
    \mathbf{D} = \bm{\varepsilon}\mathbf{E},\ \bm{\varepsilon} = 
    \begin{pmatrix}
        \varepsilon_{11} & \varepsilon_{12} & \varepsilon_{13} \\
        \varepsilon_{21} & \varepsilon_{22} & \varepsilon_{23} \\
        \varepsilon_{31} & \varepsilon_{32} & \varepsilon_{33} \\
    \end{pmatrix}.
    $$
    四面体对称性包括三个四重反演轴,绕 x,y,z 轴旋转的操作分别记为 $\mathbf{A}_x,\mathbf{A}_y,\mathbf{A}_z$,反演操作记为 $\mathbf{I}$.
    $$
    \mathbf{A}_x = 
    \begin{pmatrix}
        1 & 0 & 0 \\
        0 & 0 & 1 \\
        0 & -1 & 0 \\
    \end{pmatrix},
    \mathbf{A}_y = 
    \begin{pmatrix}
        0 & 0 & -1 \\
        0 & 1 & 1 \\
        1 & 0 & 0 \\
    \end{pmatrix},
    \mathbf{A}_z = 
    \begin{pmatrix}
        0 & -1 & 0 \\
        1 & 0 & 0 \\
        0 & 0 & 1 \\
    \end{pmatrix},
    \mathbf{I} = 
    \begin{pmatrix}
        -1 & 0 & 0 \\
        0 & -1 & 0 \\
        0 & 0 & -1 \\
    \end{pmatrix}.
    $$
    由题意,应有 $\left(\mathbf{I}\mathbf{A}_x\right)\bm{\varepsilon}\left(\mathbf{I}\mathbf{A}_x\right)^T = \bm{\varepsilon}$.即
    $$
    \begin{pmatrix}
        -1 & 0 & 0 \\
        0 & 0 & -1 \\
        0 & 1 & 0 \\
    \end{pmatrix}
    \begin{pmatrix}
        \varepsilon_{11} & \varepsilon_{12} & \varepsilon_{13} \\
        \varepsilon_{21} & \varepsilon_{22} & \varepsilon_{23} \\
        \varepsilon_{31} & \varepsilon_{32} & \varepsilon_{33} \\
    \end{pmatrix}
    \begin{pmatrix}
        -1 & 0 & 0 \\
        0 & 0 & 1 \\
        0 & -1 & 0 \\
    \end{pmatrix} = 
    \begin{pmatrix}
        \varepsilon_{11} & \varepsilon_{12} & \varepsilon_{13} \\
        \varepsilon_{21} & \varepsilon_{22} & \varepsilon_{23} \\
        \varepsilon_{31} & \varepsilon_{32} & \varepsilon_{33} \\
    \end{pmatrix}
    $$
    可得 $\varepsilon_{13} = \varepsilon_{12} = 0,\varepsilon_{21} = -\varepsilon_{31},\varepsilon_{32} = -\varepsilon_{23},\varepsilon_{22} = \varepsilon_{33}$. \\
    利用 $\left(\mathbf{I}\mathbf{A}_y\right)\bm{\varepsilon}\left(\mathbf{I}\mathbf{A}_y\right)^T = \bm{\varepsilon}$ 及  $\left(\mathbf{I}\mathbf{A}_z\right)\bm{\varepsilon}\left(\mathbf{I}\mathbf{A}_z\right)^T = \bm{\varepsilon}$ 可知 
    $\bm{\varepsilon}_{\alpha\beta} = \varepsilon_0\delta_{\alpha\beta}$.
\end{solution}

\begin{problem}
    若 $AB_3$ 的立方结构如图所示,设 $A$ 原子的散射因子为 $f_A(\mathbf{K}_{hkl})$,
    $B$ 原子的散射因子 $f_B(\mathbf{K}_{hkl})$.
    \begin{enumerate}[(a)]
        \item 求其几何结构因子 $F(\mathbf{K}_{hkl}) = ?$
        \item 找出 $\left(hkl\right)$ 衍射面的 X射线衍射强度分别在什么情况下有
              $$ 
              I\left(\mathbf{K}_{hkl}\right)\propto\begin{cases}
              \left|f_A(\mathbf{K}_{hkl}) + 3f_B(\mathbf{K}_{hkl})\right|^2 \\
              \left|f_A(\mathbf{K}_{hkl}) - f_B(\mathbf{K}_{hkl})\right|^2
              \end{cases}
              $$
        \item 设 $f_A(\mathbf{K}_{hkl}) = f_B(\mathbf{K}_{hkl})$,问衍射面指数中哪些反射消失?试举出五种最简单的。
    \end{enumerate}
\end{problem}
\begin{solution}
    \begin{enumerate}[(a)]
        \item 取原子坐标 A $\left(0,0,0\right)$ , B $\left(\dfrac{1}{2},\dfrac{1}{2},0\right),\left(\dfrac{1}{2},0,\dfrac{1}{2}\right),\left(0,\dfrac{1}{2},\dfrac{1}{2}\right)$. \\[12pt]
              $\displaystyle F(hkl) = \sum_{j}f_j\text{e}^{-2\pi i\left(hr_{j1} + kr_{j2} + lr_{j3}\right)} = f_A + f_B\left(\text{e}^{-\pi i\left(h+k\right)} + \text{e}^{-\pi i\left(k+l\right)} + \text{e}^{-\pi i\left(h+l\right)}\right)$.
        \item 当 $\left(h+k\right),\left(h+l\right),\left(k+l\right)$ 均为偶数时,$F(hkl) = f_A + 3f_B$,\ $I\left(\mathbf{K}_{hkl}\right)\propto\left|f_A(\mathbf{K}_{hkl}) + 3f_B(\mathbf{K}_{hkl})\right|^2$. \\[12pt]
              当 $\left(h+k\right),\left(h+l\right),\left(k+l\right)$ 两奇一偶时,$F(hkl) = f_A - f_B$,\ $I\left(\mathbf{K}_{hkl}\right)\propto\left|f_A(\mathbf{K}_{hkl}) - f_B(\mathbf{K}_{hkl})\right|^2$.
        \item 消光条件 $F(hkl) = 0$,据此可得 $\left(1,0,0\right),\left(0,1,0\right),\left(0,0,1\right),\left(1,1,0\right),\left(1,0,1\right)$.
    \end{enumerate}
\end{solution}

\begin{problem}
    在某立方晶系的铜 $\mathbf{K}_\alpha X$ 射线粉末相中,观察到的衍射角 $\theta_i$ 有下列关系:
    $$\sin{\theta_1} : \sin{\theta_2} : \sin{\theta_3} : \sin{\theta_4} 
    : \sin{\theta_5}: \sin{\theta_6}: \sin{\theta_7}: \sin{\theta_8}$$
    $$= \sqrt{3} : \sqrt{4} : \sqrt{8} : \sqrt{11} : \sqrt{12} : \sqrt{16} : \sqrt{19} : \sqrt{20}.$$
    \begin{enumerate}[(a)]
        \item 试确定对应于这些衍射角的晶面的衍射面指数;
        \item 问该立方晶体时简单立方、面心立方还是体心立方?
    \end{enumerate}
\end{problem}
\begin{solution}
    \begin{enumerate}[(a)]
        \item 晶面间距 $d_{hkl} = \dfrac{a}{\sqrt{h^2+k^2+l^2}}$,布拉格反射定律 $2d_{hkl}\sin{\theta} = n\lambda$, \\[12pt]
              可得 $\sin{\theta}\propto\sqrt{\left(nh\right)^2 + \left(nk\right)^2 + \left(nl\right)^2}$. \\[12pt]
              故衍射面指数 $\left(1,1,1\right),\left(2,0,0\right),\left(2,2,0\right),\left(1,1,3\right),\left(2,2,2\right),\left(4,0,0\right),\left(3,3,1\right),\left(4,2,1\right)$.
        \item 简单立方允许所有 $\left(hkl\right)$ 值,没有消光. \\
              体心立方要求 $\left(h+k+l\right)$ 为偶数. \\
              面心立方则要求 $h,k,l$ 全奇或全偶. \\
              故该立方晶体是面心立方。
    \end{enumerate}
\end{solution}

\begin{problem}
    X 射线衍射的线宽。 \\
    假定一个有限大小的晶体,点阵节点由 $\displaystyle R_l = \sum_{i = 1}^{3}l_i\mathbf{a}_i$ 确定,
    其中 $l_i$ 取整数 $0,1,2,\cdots,N_i - 1$,每个结点处有全同的点散射中心。散射振幅可写为
    $$u_{\mathbf{k}\to \mathbf{k}^\prime} = c\sum_{l_i = 0}^{N_i - 1}\text{e}^{-i\left(\mathbf{k}^\prime - \mathbf{k}\right)\cdot\sum\limits_{i = 1}^{3}l_i\mathbf{a}_i}.$$
    \begin{enumerate}[(a)]
        \item 证明散射强度 $\displaystyle I = \left|u\right|^2 = u^*u = c^2\prod_{i = 1}^{3}
              \dfrac{\sin^2{\dfrac{1}{2}N_i\left(\Delta\mathbf{k}\cdot\mathbf{a}_i\right)}}{\sin^2{\dfrac{1}{2}\left(\Delta\mathbf{k}\cdot\mathbf{a}_i\right)}},\ \Delta k = k^\prime - k$;
        \item 当 $\Delta\mathbf{k}\cdot\mathbf{a}_i = 2\pi h_i$($h_i$ 为整数)时,出现衍射极大值,函数 $\sin^2{\dfrac{1}{2}N_i\left(\Delta\mathbf{k}\cdot\mathbf{a}_i\right)}$ 的第一个零点定义了 X 射线衍射的线宽
              $\Delta_i$,证明 $\Delta_i = \dfrac{2\pi}{N_i}$;
        \item 对于一个无限大的晶体,$\displaystyle N_i\to\infty$,证明 $\displaystyle I = c^2N^2\delta_{\mathbf{k}^\prime - \mathbf{k},\mathbf{K}_h}$.
    \end{enumerate}
\end{problem}
\begin{solution}
    \begin{enumerate}[(a)]
        \item 对散射振幅分析,$\displaystyle u_{\mathbf{k}\to \mathbf{k}^\prime} = c\sum_{l_i = 0}^{N_i - 1}\text{e}^{-i\left(\mathbf{k}^\prime - \mathbf{k}\right)\cdot\sum\limits_{i = 1}^{3}l_i\mathbf{a}_i} 
              = c\sum_{l_i = 0}^{N_i - 1}\text{e}^{-il_1\left(\Delta\mathbf{k}\cdot\mathbf{a}_1\right)}\cdot\text{e}^{-il_2\left(\Delta\mathbf{k}\cdot\mathbf{a}_2\right)}\cdot\text{e}^{-il_3\left(\Delta\mathbf{k}\cdot\mathbf{a}_3\right)}$. \\[12pt]
              写成连乘形式 $\displaystyle u_{\mathbf{k}\to \mathbf{k}^\prime} = c\prod_{i = 1}^{3}\sum_{l_i = 0}^{N_i - 1}\text{e}^{-il_i\left(\Delta\mathbf{k}\cdot\mathbf{a}_i\right)}$. \\[12pt]
              $\displaystyle \sum_{l_i = 0}^{N_i - 1}\text{e}^{-il_i\left(\Delta\mathbf{k}\cdot\mathbf{a}_i\right)} = \frac{1 - \text{e}^{-i N_i \left(\Delta\mathbf{k} \cdot \mathbf{a}_i\right)}}{1 - \text{e}^{-i\left(\Delta\mathbf{k}\cdot\mathbf{a}_i\right)}}
              = \frac{\text{e}^{-i\frac{1}{2}N_i \left(\Delta\mathbf{k} \cdot \mathbf{a}_i\right)}\left(\text{e}^{i\frac{1}{2}N_i \left(\Delta\mathbf{k} \cdot \mathbf{a}_i\right)} - \text{e}^{-i\frac{1}{2}N_i \left(\Delta\mathbf{k} \cdot \mathbf{a}_i\right)}\right)}{\text{e}^{-i\frac{1}{2}\left(\Delta\mathbf{k} \cdot \mathbf{a}_i\right)}\left(\text{e}^{i\frac{1}{2}\left(\Delta\mathbf{k} \cdot \mathbf{a}_i\right)} - \text{e}^{-i\frac{1}{2}\left(\Delta\mathbf{k} \cdot \mathbf{a}_i\right)}\right)}$. \\[12pt]
              由欧拉公式可化简为 $\displaystyle \sum_{l_i = 0}^{N_i - 1}\text{e}^{-il_i\left(\Delta\mathbf{k}\cdot\mathbf{a}_i\right)} = \frac{\text{e}^{-i\frac{1}{2}N_i \left(\Delta\mathbf{k} \cdot \mathbf{a}_i\right)}\sin{\frac{1}{2}N_i \left(\Delta\mathbf{k} \cdot \mathbf{a}_i\right)}}{\text{e}^{-i\frac{1}{2} \left(\Delta\mathbf{k} \cdot \mathbf{a}_i\right)}\sin{\frac{1}{2} \left(\Delta\mathbf{k} \cdot \mathbf{a}_i\right)}}$. \\[12pt]
              故 $\displaystyle I = \left|u\right|^2 = u^*u = c^2\prod_{i = 1}^{3}\dfrac{\sin^2{\dfrac{1}{2}N_i\left(\Delta\mathbf{k}\cdot\mathbf{a}_i\right)}}{\sin^2{\dfrac{1}{2}\left(\Delta\mathbf{k}\cdot\mathbf{a}_i\right)}},\ \Delta k = k^\prime - k$.
        \item 函数 $\sin^2{\dfrac{1}{2}N_i\left(\Delta\mathbf{k}\cdot\mathbf{a}_i\right)}$ 的第一个零点出现在:$\displaystyle\frac{1}{2} N_i (\Delta \mathbf{k} \cdot \mathbf{a}_i) = \pi \implies \Delta \mathbf{k} \cdot \mathbf{a}_i = \frac{2\pi}{N_i}$. \\[12pt]
              即 $\Delta_i = \dfrac{2\pi}{N_i}$.
        \item \textcolor{red}{当 $N_i\to\infty$ 时,每个求和式 $\displaystyle\sum_{l_i=0}^{N_i-1}\text{e}^{\left(-i l_i(\Delta\mathbf{k}\cdot\mathbf{a}_i)\right)} $ 转换为 $\delta$ 函数。因此,散射强度 $I$ 表现为 $\delta$ 函数的形式:
              $$I = c^2 N^2 \delta_{\mathbf{k}' - \mathbf{k}, \mathbf{K}_h}$$
              其中 $\mathbf{K}_h$ 是倒格矢,满足布拉格条件。}
    \end{enumerate}
\end{solution}
\begin{note}
    \begin{itemize}
        \item (c)不是很理解。
    \end{itemize}
\end{note}

\newpage
\section{第二章习题}

\begin{problem}
    导出 NaCl 型离子晶体中排斥势指数的下列关系式:
    $$n = 1 + \frac{4\pi\varepsilon_0\times18Br_0^4}{\alpha\text{e}^2}\left(\text{SI单位}\right)$$
    其中 $r_0$ 为近邻离子间距,$\alpha$ 为以 $r_0$ 为单位的马德隆常数,$B$ 为体积弹性模量。
    已知 NaCl晶体的 $B = 2.4\times10^{10}\text{N}/\text{m}^2$, $r_0 = 2.81\mathring{A}$,求 NaCl 的 $n = ?$
\end{problem}
\begin{solution}
    $\alpha = 1.747558$, $\varepsilon = 8.854\times10^{-12}$. 代入公式有
    $$n = 1 + \frac{4\pi\varepsilon_0\times18Br_0^4}{\alpha\text{e}^2}
    = 1 + \frac{4\pi\times8.854\times10^{-12}\times18\times2.4\times10^{10}\times\left(2.81\times10^{-10}\right)^4}
    {1.747588\times\left(1.60219\times10^{-19}\right)} = 7.78$$
\end{solution}
\begin{note}
    见书 P63,P64。 \\
    设晶体有 $N$ 个元胞,晶体内能 $U = N\left(-\dfrac{A_1}{r} + \dfrac{A_n}{r}\right), A_1 = \dfrac{\alpha e^2}{4\pi\varepsilon_0}, A_n = 6b, V = 2Nr^3$. \\[12pt]
    由平衡位置能量极小有 $\dfrac{\text{d}U(r)}{\text{d}r} = N\left(\dfrac{\alpha e^2}{4\pi\varepsilon_0r^2} - \dfrac{6nb}{r^{n+1}}\right) = 0$,
    即 $6b = \dfrac{\alpha e^2r_0^{n-1}}{4n\pi\varepsilon_0}$. \\[12pt]
    代回内能公式有 $U = N\dfrac{\alpha e^2}{4\pi\varepsilon_0}\left(-\dfrac{1}{r} + \dfrac{r_0^{n-1}}{nr^{n}}\right)$. \\[12pt]
    体积弹性模量 $B = V\dfrac{\text{d}^2U}{\text{d}V^2}\Big|_{r_0} = \dfrac{1}{18Nr_0}\dfrac{\text{d}^2}{\text{d}r^2}\Big|_{r_0} = \dfrac{(n-1)\alpha e^2}{4\pi\varepsilon_0\times18r_0^4}$. \\[12pt]
    晶体结合能 $W = -U(r_0) = \dfrac{1}{4\pi\varepsilon}\dfrac{N\alpha e^2}{r_0}\left(1 - \dfrac{1}{n}\right)$.
\end{note}

\begin{problem}
    带 $\pm e$ 电荷的两种离子相间排成一维晶格,设 N 为元胞数,$\dfrac{A_n}{r_0^n}$ 为排斥势,$r_0$ 为正负离子间距。求证,当 N 很大时有:
    \begin{enumerate}[(a)]
        \item 马德隆常数 $\alpha = 2\ln{2}$;
        \item 结合能 $W = \dfrac{Ne^2 2\ln{2}}{4\pi\varepsilon_0r_0}\left(1-\dfrac{1}{n}\right)$;
        \item 当压缩晶格时 $r\to r_0\left(1-\delta\right)$,且 $\delta\ll1$,则需做功 $\dfrac{1}{2}\left(2Nr_0\right)B\delta^2$,其中线弹模
              $$B = \frac{\left(n-1\right)N 2\ln{2}}{8\pi\varepsilon_0r_0^2}e^2$$
    \end{enumerate}
\end{problem}
\begin{solution}
    \begin{enumerate}[(a)]
        \item 取一个电子分析,其静电势能 $\displaystyle u = 2\times\left(-\dfrac{e^2}{4\pi\varepsilon_0}\sum_{i=1}^{\infty}\frac{(-)^i}{ir_0}\right)$. \\[12pt]
              而 $\displaystyle\ln{(1+x)} = \sum_{n=1}^{\infty}\frac{(-)^{n-1}}{n}x$, 取 $x=1$ 得 $\alpha = 2\ln2$.
        \item 内能还需考虑排斥势能,在单电子分析上乘元胞数 $N$。 \\[12pt]
              同习题9注记,此时 $\alpha = 2\ln{2}$,可得 $W = \dfrac{Ne^2 2\ln{2}}{4\pi\varepsilon_0r_0}\left(1-\dfrac{1}{n}\right)$.
        \item 将内能函数展开,一次项由于平衡位置能量极小为零,故在 $\delta\to0$时仅需考虑二次项变化。 \\[12pt]
              $U(r) \approx U(r_0) + \dfrac{1}{2}\cdot\dfrac{\text{d}^2 U}{\text{d}r^2}\Big|_{r=r_0}(r-r_0)^2 = \dfrac{\alpha N e^2(n-1)}{8\pi\varepsilon_0r_0^3}(r-r_0)^2$. \\[12pt]
              做功等于内能增量 $W = U(r_0(1-\delta)) - U(r_0) = \dfrac{\alpha N e^2(n-1)}{8\pi\varepsilon_0r_0^3}r_0^2\delta^2 = \dfrac{1}{2}(2Nr_0)B\delta^2$. \\[12pt]
              故 $\displaystyle B = \frac{\left(n-1\right)N 2\ln{2}}{8\pi\varepsilon_0r_0^2}e^2$.
    \end{enumerate}
\end{solution}

\begin{problem}
    量子固体。 \\
    在量子固体中,起主导作用的排斥能是原子的零点振动能,考虑晶态 $^4\text{He}$ 的一个粗略一维模型,
    即每个氦原子局限在一段长为 L 的线段上,每段内的基态波函数取为半波长为 L 的自由粒子波函数。
    \begin{enumerate}[(a)]
        \item 试求每个粒子的零点振动能;
        \item 推导维持该线段不发生膨胀所需力的表达式;
        \item 在平衡时,动能所引起的膨胀倾向被范德瓦耳斯相互作用所平衡,假定最近邻间的范德瓦耳斯能为 $U(L) = 1.6L^{-6}10^{-79}J$,其中 L 以 m 为单位,求 L 的平衡值。
    \end{enumerate}
\end{problem}
\begin{solution}
    \begin{enumerate}[(a)]
        \item 波长 $\lambda = 2L$, 有 $p = \dfrac{h}{\lambda} = \dfrac{h}{2L}$, $T = \dfrac{p^2}{2m} = \dfrac{h^2}{8mL^2}$.
        \item 记 $U(L)$ 为吸引势,总能量 $E(L) = U(L) + T(L) = U(L) + \dfrac{h^2}{8mL^2}$. \\[12pt]
              平衡位置有 $\dfrac{\text{d}E(L)}{\text{d}L} = \dfrac{\text{d}U(L)}{\text{d}L} - \dfrac{h^2}{4mL^3} = 0$. \\[12pt]
              故不发生膨胀所需力为 $F(L) = -\dfrac{\text{d}U(L)}{\text{d}L} = -\dfrac{h^2}{4mL^3}$.
        \item 平衡位置有 $\dfrac{\text{d}E(L)}{\text{d}L} = \dfrac{\text{d}U(L)}{\text{d}L} - \dfrac{h^2}{4mL^3} = 0$,代入 $U(L) = 1.6L^{-6}10^{-79}J$ 得 \\[12pt]
              $L = \left(\dfrac{4m\times9.6\times10^{-79}}{h^2}\right)^{\frac{1}{4}} = \left(\dfrac{4\times1.67\times10^{-27}\times9.6\times10^{-79}}{6.626\times10^{-34}}\right)^{\frac{1}{4}} = 4.918\times10^{-10}\text{m} = 4.918\mathring{A}$. 
    \end{enumerate}
\end{solution}
\begin{note}
    \begin{itemize}
        \item (a)基态波函数满足一维无限深势阱条件:$\displaystyle\Psi_0(x) = \sqrt{\frac{2}{L}} \sin\left(\frac{\pi x}{L}\right)$,
              动能算符 $\displaystyle\hat{T} = -\frac{\hbar^2}{2m}\frac{\text{d}^2}{\text{d}x^2}$. \\[12pt]
              零点振动能为动能期望值$$E_T = \langle\Psi_0|\hat{T}|\Psi_0\rangle
              = \frac{\hbar^2}{2m}\int_0^L\left|\frac{d\Psi_0}{dx}\right|^2\text{d}x
              = \frac{\hbar^2\pi^2}{2mL^2}\cdot\frac{1}{L}\int_0^L \sin^2\left(\frac{\pi x}{L}\right)\text{d}x
              = \frac{\hbar^2\pi^2}{4mL^2}.$$
    \end{itemize}
\end{note}

\end{document}
